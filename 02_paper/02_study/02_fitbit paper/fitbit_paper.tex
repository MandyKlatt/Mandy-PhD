% Options for packages loaded elsewhere
\PassOptionsToPackage{unicode}{hyperref}
\PassOptionsToPackage{hyphens}{url}
%
\documentclass[
  man]{apa6}
\usepackage{amsmath,amssymb}
\usepackage{lmodern}
\usepackage{iftex}
\ifPDFTeX
  \usepackage[T1]{fontenc}
  \usepackage[utf8]{inputenc}
  \usepackage{textcomp} % provide euro and other symbols
\else % if luatex or xetex
  \usepackage{unicode-math}
  \defaultfontfeatures{Scale=MatchLowercase}
  \defaultfontfeatures[\rmfamily]{Ligatures=TeX,Scale=1}
\fi
% Use upquote if available, for straight quotes in verbatim environments
\IfFileExists{upquote.sty}{\usepackage{upquote}}{}
\IfFileExists{microtype.sty}{% use microtype if available
  \usepackage[]{microtype}
  \UseMicrotypeSet[protrusion]{basicmath} % disable protrusion for tt fonts
}{}
\makeatletter
\@ifundefined{KOMAClassName}{% if non-KOMA class
  \IfFileExists{parskip.sty}{%
    \usepackage{parskip}
  }{% else
    \setlength{\parindent}{0pt}
    \setlength{\parskip}{6pt plus 2pt minus 1pt}}
}{% if KOMA class
  \KOMAoptions{parskip=half}}
\makeatother
\usepackage{xcolor}
\usepackage{graphicx}
\makeatletter
\def\maxwidth{\ifdim\Gin@nat@width>\linewidth\linewidth\else\Gin@nat@width\fi}
\def\maxheight{\ifdim\Gin@nat@height>\textheight\textheight\else\Gin@nat@height\fi}
\makeatother
% Scale images if necessary, so that they will not overflow the page
% margins by default, and it is still possible to overwrite the defaults
% using explicit options in \includegraphics[width, height, ...]{}
\setkeys{Gin}{width=\maxwidth,height=\maxheight,keepaspectratio}
% Set default figure placement to htbp
\makeatletter
\def\fps@figure{htbp}
\makeatother
\setlength{\emergencystretch}{3em} % prevent overfull lines
\providecommand{\tightlist}{%
  \setlength{\itemsep}{0pt}\setlength{\parskip}{0pt}}
\setcounter{secnumdepth}{-\maxdimen} % remove section numbering
% Make \paragraph and \subparagraph free-standing
\ifx\paragraph\undefined\else
  \let\oldparagraph\paragraph
  \renewcommand{\paragraph}[1]{\oldparagraph{#1}\mbox{}}
\fi
\ifx\subparagraph\undefined\else
  \let\oldsubparagraph\subparagraph
  \renewcommand{\subparagraph}[1]{\oldsubparagraph{#1}\mbox{}}
\fi
\newlength{\cslhangindent}
\setlength{\cslhangindent}{1.5em}
\newlength{\csllabelwidth}
\setlength{\csllabelwidth}{3em}
\newlength{\cslentryspacingunit} % times entry-spacing
\setlength{\cslentryspacingunit}{\parskip}
\newenvironment{CSLReferences}[2] % #1 hanging-ident, #2 entry spacing
 {% don't indent paragraphs
  \setlength{\parindent}{0pt}
  % turn on hanging indent if param 1 is 1
  \ifodd #1
  \let\oldpar\par
  \def\par{\hangindent=\cslhangindent\oldpar}
  \fi
  % set entry spacing
  \setlength{\parskip}{#2\cslentryspacingunit}
 }%
 {}
\usepackage{calc}
\newcommand{\CSLBlock}[1]{#1\hfill\break}
\newcommand{\CSLLeftMargin}[1]{\parbox[t]{\csllabelwidth}{#1}}
\newcommand{\CSLRightInline}[1]{\parbox[t]{\linewidth - \csllabelwidth}{#1}\break}
\newcommand{\CSLIndent}[1]{\hspace{\cslhangindent}#1}
\ifLuaTeX
\usepackage[bidi=basic]{babel}
\else
\usepackage[bidi=default]{babel}
\fi
\babelprovide[main,import]{english}
% get rid of language-specific shorthands (see #6817):
\let\LanguageShortHands\languageshorthands
\def\languageshorthands#1{}
% Manuscript styling
\usepackage{upgreek}
\captionsetup{font=singlespacing,justification=justified}

% Table formatting
\usepackage{longtable}
\usepackage{lscape}
% \usepackage[counterclockwise]{rotating}   % Landscape page setup for large tables
\usepackage{multirow}		% Table styling
\usepackage{tabularx}		% Control Column width
\usepackage[flushleft]{threeparttable}	% Allows for three part tables with a specified notes section
\usepackage{threeparttablex}            % Lets threeparttable work with longtable

% Create new environments so endfloat can handle them
% \newenvironment{ltable}
%   {\begin{landscape}\centering\begin{threeparttable}}
%   {\end{threeparttable}\end{landscape}}
\newenvironment{lltable}{\begin{landscape}\centering\begin{ThreePartTable}}{\end{ThreePartTable}\end{landscape}}

% Enables adjusting longtable caption width to table width
% Solution found at http://golatex.de/longtable-mit-caption-so-breit-wie-die-tabelle-t15767.html
\makeatletter
\newcommand\LastLTentrywidth{1em}
\newlength\longtablewidth
\setlength{\longtablewidth}{1in}
\newcommand{\getlongtablewidth}{\begingroup \ifcsname LT@\roman{LT@tables}\endcsname \global\longtablewidth=0pt \renewcommand{\LT@entry}[2]{\global\advance\longtablewidth by ##2\relax\gdef\LastLTentrywidth{##2}}\@nameuse{LT@\roman{LT@tables}} \fi \endgroup}

% \setlength{\parindent}{0.5in}
% \setlength{\parskip}{0pt plus 0pt minus 0pt}

% Overwrite redefinition of paragraph and subparagraph by the default LaTeX template
% See https://github.com/crsh/papaja/issues/292
\makeatletter
\renewcommand{\paragraph}{\@startsection{paragraph}{4}{\parindent}%
  {0\baselineskip \@plus 0.2ex \@minus 0.2ex}%
  {-1em}%
  {\normalfont\normalsize\bfseries\itshape\typesectitle}}

\renewcommand{\subparagraph}[1]{\@startsection{subparagraph}{5}{1em}%
  {0\baselineskip \@plus 0.2ex \@minus 0.2ex}%
  {-\z@\relax}%
  {\normalfont\normalsize\itshape\hspace{\parindent}{#1}\textit{\addperi}}{\relax}}
\makeatother

% \usepackage{etoolbox}
\makeatletter
\patchcmd{\HyOrg@maketitle}
  {\section{\normalfont\normalsize\abstractname}}
  {\section*{\normalfont\normalsize\abstractname}}
  {}{\typeout{Failed to patch abstract.}}
\patchcmd{\HyOrg@maketitle}
  {\section{\protect\normalfont{\@title}}}
  {\section*{\protect\normalfont{\@title}}}
  {}{\typeout{Failed to patch title.}}
\makeatother

\usepackage{xpatch}
\makeatletter
\xapptocmd\appendix
  {\xapptocmd\section
    {\addcontentsline{toc}{section}{\appendixname\ifoneappendix\else~\theappendix\fi\\: #1}}
    {}{\InnerPatchFailed}%
  }
{}{\PatchFailed}
\keywords{heart rate; photoplethysmography; wearable electronic device; expertise differences \newline\indent Word count: X}
\DeclareDelayedFloatFlavor{ThreePartTable}{table}
\DeclareDelayedFloatFlavor{lltable}{table}
\DeclareDelayedFloatFlavor*{longtable}{table}
\makeatletter
\renewcommand{\efloat@iwrite}[1]{\immediate\expandafter\protected@write\csname efloat@post#1\endcsname{}}
\makeatother
\usepackage{lineno}

\linenumbers
\usepackage{csquotes}
\ifLuaTeX
  \usepackage{selnolig}  % disable illegal ligatures
\fi
\IfFileExists{bookmark.sty}{\usepackage{bookmark}}{\usepackage{hyperref}}
\IfFileExists{xurl.sty}{\usepackage{xurl}}{} % add URL line breaks if available
\urlstyle{same} % disable monospaced font for URLs
\hypersetup{
  pdftitle={New approaches to teachers' experience of stress: Do heart rate measurements with fitness trackers provide an efficient, inexpensive, and robust measurement method?},
  pdfauthor={Mandy Klatt1, Peer Keßler1, Gregor Kachel1,2, Christin Lotz1, \& Anne Deiglmayr1},
  pdflang={en-EN},
  pdfkeywords={heart rate; photoplethysmography; wearable electronic device; expertise differences},
  hidelinks,
  pdfcreator={LaTeX via pandoc}}

\title{New approaches to teachers' experience of stress: Do heart rate measurements with fitness trackers provide an efficient, inexpensive, and robust measurement method?}
\author{Mandy Klatt\textsuperscript{1}, Peer Keßler\textsuperscript{1}, Gregor Kachel\textsuperscript{1,2}, Christin Lotz\textsuperscript{1}, \& Anne Deiglmayr\textsuperscript{1}}
\date{}


\shorttitle{Measuring novice and expert teachers' heart rate}

\authornote{

We received funding from QualiFond of University Leipzig. We have no conflicts of interest to disclose. This article is based on data used at conference presentations (DACH-Nachwuchsakademie, 2022; EARLI SIG 11, 2022; EARLI SIG 27, 2022).

The authors made the following contributions. Mandy Klatt: Conceptualization, Writing - Original Draft Preparation, Writing - Review \& Editing; Peer Keßler: Writing - Original Draft Preparation, Writing - Review \& Editing; Gregor Kachel: Conceptualization, Writing - Original Draft Preparation, Writing - Review \& Editing; Christin Lotz: Conceptualization, Writing - Original Draft Preparation, Writing - Review \& Editing; Anne Deiglmayr: Supervision.

Correspondence concerning this article should be addressed to Mandy Klatt, Dittrichring 5-7, 04109 Leipzig. E-mail: \href{mailto:mandy.klatt@uni-leipzig.de}{\nolinkurl{mandy.klatt@uni-leipzig.de}}

}

\affiliation{\vspace{0.5cm}\textsuperscript{1} Leipzig University\\\textsuperscript{2} Max Planck Institute for Evolutionary Anthropology}

\abstract{%
One or two sentences providing a \textbf{basic introduction} to the field, comprehensible to a scientist in any discipline.

Two to three sentences of \textbf{more detailed background}, comprehensible to scientists in related disciplines.

One sentence clearly stating the \textbf{general problem} being addressed by this particular study.

One sentence summarizing the main result (with the words ``\textbf{here we show}'' or their equivalent).

Two or three sentences explaining what the \textbf{main result} reveals in direct comparison to what was thought to be the case previously, or how the main result adds to previous knowledge.

One or two sentences to put the results into a more \textbf{general context}.

Two or three sentences to provide a \textbf{broader perspective}, readily comprehensible to a scientist in any discipline.
}



\begin{document}
\maketitle

\hypertarget{introduction}{%
\section{Introduction}\label{introduction}}

Prospective teachers experience the first five years as a particular professional challenge, as they are associated with a high workload (e.g., class size, poor student behavior) and resulting stress in the teaching profession (Struyven \& Vanthournout, 2014).

Psychosomatic stress symptoms, such as increasing heart rate, result from an interaction of situational stressors and person-specific available resources {[}``transactional stress model''; Lazarus and Folkman (1987); Obbarius, Fischer, Liegl, Obbarius, and Rose (2021){]}. If teachers are exposed to a stressful teaching-learning environment for a long time and do not have the sufficient resources and coping strategies, this can lead to burnout.

Fisher (2011) showed in her study of 400 secondary school teachers that burnout scores differed significantly between novice and experienced teachers, with novices having higher burnout scores. In addition, she found that a low number of years of professional experience was a statistically significant predictor of stress in the teaching profession, along with job satisfaction and burnout.

In addition to self-reported indicators of stress (e.g., emotional exhaustion), psychophysiological indicators (e.g., heart rate) are becoming increasingly important in research. Heart rate as the number of heart contractions per minute (Löllgen, 2015) is an important indicator of physical or emotional stress. Increased workload is associated with increased heart rate (Sachs, 2014). Arguably, physiological data provide a more objective and non-reactive measure of stress in the classroom (Runge et al., 2020). However, in an educational context, this requires the use of low-cost, non-invasive instruments. Fitness trackers worn on the wrist could fill this research gap (Ferguson, Rowlands, Olds, \& Maher, 2015).

\hypertarget{aim-of-the-study}{%
\subsection{Aim of the study}\label{aim-of-the-study}}

The aim of the study was to uncover differences between 25 experienced and 25 inexperienced teachers on the physical and emotional demands during a 15-minute micro-teaching unit by measuring their heart rate using a fitness watch (Fitbit Charge 4). As the fitness watch was put on at the beginning of the survey and was not taken off until the entire data collection was completed, the heart rate could be measured in the resting and arousal states. In addition, the watch recorded the subjects' movement activity to determine how many steps were taken during the lesson. The practicality of the FitBit for classroom research is discussed.

\hypertarget{methods}{%
\section{Methods}\label{methods}}

We report how we determined our sample size, all data exclusions (if any), all manipulations, and all measures in the study.

\hypertarget{participants}{%
\subsection{Participants}\label{participants}}

XXX fitbit data exclusion XXX

The sample consisted of \emph{N} = 41 participants with \emph{n} = 16 expert teachers and \emph{n} = 25 novice teachers.

The inclusion criterion for experts was that they have successfully completed teacher training and are actively employed in the teaching profession. According to Palmer, Stough, Burdenski, and Gonzales (2005), we selected teachers as experts who had at least three years of professional experience and ideally had worked in another teaching position, such as subject advisor or trainer for trainee teachers, in addition to their teaching profession in school. Novices were student teachers who had successfully completed their first internship in a school and gained one to four hours of teaching experience.

The expert teachers (9 women; 56.25\%) had a mean age of 41.40 years (\emph{SD} = 10.70; range: 27-59) and an average teaching experience of 13.60 years (\emph{SD} = 12.80; range: 2-37).
18.75\% of the experts were primary school teachers and 81.25\% were secondary school teachers. 56\% of the experienced teachers were also engaged in an secondary teaching activity, such as lecturers at the university, main training supervisors for trainee teachers and subject advisers.

The novice teachers (16 women; 64\%) had a mean age of 23.40 years (\emph{SD} = 1.80; range: 20-27) with an average teaching experience of 0 years. On average, the student teachers were in their 7.30 semester (\emph{SD} = 2.60; range: 3-11). Furthermore, they had an average teaching experience of 11.20 teaching units à 45min (\emph{SD} = 8.10; range: 0-36) through the internships during their studies.
20\% of the novices were studying to become primary school teachers, 68\% to become secondary school teachers and 12\% to become special education teachers. 88\% of the student teachers were also engaged in an extracurricular teaching activity, such as tutoring or homework supervision.

The subjects were primarily recruited through personal contacts, social media (Facebook), e-mail distribution lists and advertising in lectures at Leipzig University. All study procedures were carried out in accordance with the ethical standards of the University's Institutional Review Board. The authors received a positive vote on the study procedures from the Ethics Committee Board of Leipzig University. All participants were informed in detail about the aim and intention of the study prior to testing. Participation in the study was voluntary and only took place after written consent has been given.

\hypertarget{material}{%
\subsection{Material}\label{material}}

We used a Fitbit Charge 4 to measure the teachers' heart rate. The device was attached 2-finger widths above the ulnar styloid process to the subject's wrist following the manufacturer's instructions. To determine the heart rate, the Fitbit flashes green LEDs many times per second and uses light-sensitive photodiodes to measure the volume changes in the capillaries and then calculates how many times the heart beats per minute (bpm). Data were automatically wireless synced with an iPad via Bluetooth to a Fitbit account, and subsequently, the intraday second-by-second data were exported for each session using the opensource software Pulse Watch (PulseWatch. URL: \url{https://iccir919.github.io/pulseWatch/public/index.html} {[}accessed 2022-08-03{]}).

\hypertarget{procedure}{%
\subsection{Procedure}\label{procedure}}

In a laboratory setting, three trained actors performed teaching disruptions in a counter balanced fashion while the subject taught a 15-minute micro-teaching unit prepared in advance. To record the subject's heart rate, the Fitbit Charge 4 was put on at least 10 minutes before the start of the lesson. The lesson was recorded by four cameras and an audio recorder. In addition, the subject wore eye-tracking glasses to record gaze behavior.

Subsequently, the subjects as well as the actors were given a short questionnaire, which contained items to collect demographic information as well as items about the previously given lesson on teaching quality using a validated questionnaire (EMU, Helmke et al. (2014)) and self developed scales on the teacher's presence behavior derived from the research literature via the online survey website SoSci Survey (4-point Likert scale; 1 = strongly disagree; 2 = disagree; 3 = agree; 4 = strongly agree). The completion of the questionnaire took approximately 5 minutes.

The experimenter then conducted a Stimulated Recall Interview (SRI), where the subject commented and rated the reactions to classroom events while watching the eye tracking video.

Finally, a Situational Judgment Test (SJT, Gold and Holodynski (2015)) was used to assess the subject's strategic knowledge of classroom management. The subject was asked to judge alternative actions on school scenarios on a 6-point rating scale from 1 (A) to 6 (F) according to school grades. Data from the SJT were again collected as an online questionnaire via the website www.soscisurvey.de and lastet approximately 10 minutes.

The Fitbit watch was removed only after the last questionnaire to obtain heart rate at three different measurement time points: before, during, and after the lesson.

\hypertarget{data-analysis}{%
\subsection{Data analysis}\label{data-analysis}}

We used R (Version 4.1.3; R Core Team, 2022) and the R-packages \emph{papaja} (Version 0.1.0.9999; Aust \& Barth, 2020), and \emph{tinylabels} (Version 0.2.3; Barth, 2022) for all our analyses.

\hypertarget{fitbit}{%
\subsubsection{FitBit}\label{fitbit}}

All participants were given a FitBit Smart Watch Charge 4 to wear during the experiment.

\hypertarget{heart-rate}{%
\subsection{Heart Rate}\label{heart-rate}}

The heart rate of each subject were measured before, during and after the experiment.

XXX Data and data wrangling need to be imported XXX

\hypertarget{steps}{%
\subsubsection{Steps}\label{steps}}

Using the FitBit we measured the steps a person walked during the experiment as well. We validated the steps in six randomly selected videos by manual step counting. A step was considered to be the movement of the subject's foot forward, backward, or sideways. The measurement was not as reliable as hoped.

\hypertarget{results}{%
\section{Results}\label{results}}

\hypertarget{discussion-and-implication}{%
\section{Discussion and implication}\label{discussion-and-implication}}

\newpage

\hypertarget{references}{%
\section{References}\label{references}}

\hypertarget{refs}{}
\begin{CSLReferences}{1}{0}
\leavevmode\vadjust pre{\hypertarget{ref-R-papaja}{}}%
Aust, F., \& Barth, M. (2020). \emph{{papaja}: {Prepare} reproducible {APA} journal articles with {R Markdown}}. Retrieved from \url{https://github.com/crsh/papaja}

\leavevmode\vadjust pre{\hypertarget{ref-R-tinylabels}{}}%
Barth, M. (2022). \emph{{tinylabels}: Lightweight variable labels}. Retrieved from \url{https://cran.r-project.org/package=tinylabels}

\leavevmode\vadjust pre{\hypertarget{ref-ferguson2015validity}{}}%
Ferguson, T., Rowlands, A. V., Olds, T., \& Maher, C. (2015). The validity of consumer-level, activity monitors in healthy adults worn in free-living conditions: A cross-sectional study. \emph{International Journal of Behavioral Nutrition and Physical Activity}, \emph{12}(1), 1--9.

\leavevmode\vadjust pre{\hypertarget{ref-gold2015development}{}}%
Gold, B., \& Holodynski, M. (2015). Development and construct validation of a situational judgment test of strategic knowledge of classroom management in elementary schools. \emph{Educational Assessment}, \emph{20}(3), 226--248.

\leavevmode\vadjust pre{\hypertarget{ref-helmke2014unterrichtsdiagnostik}{}}%
Helmke, A., Helmke, T., Lenske, G., Pham, G., Praetorius, A.-K., Schrader, F.-W., \& AdeThurow, M. (2014). Unterrichtsdiagnostik mit EMU. \emph{Aus-Und Fortbildung Der Lehrkr{ä}fte in Hinblick Auf Verbesserung Der Diagnosef{ä}higkeit, Umgang Mit Heterogenit{ä}t Und Individuelle F{ö}rderung}, 149--163.

\leavevmode\vadjust pre{\hypertarget{ref-lazarus1987transactional}{}}%
Lazarus, R. S., \& Folkman, S. (1987). Transactional theory and research on emotions and coping. \emph{European Journal of Personality}, \emph{1}(3), 141--169.

\leavevmode\vadjust pre{\hypertarget{ref-Luxf6llgen2015}{}}%
Löllgen, H. (2015). Herzfrequenz und blutdruck. In J. Niebauer (Ed.), \emph{Sportkardiologie} (pp. 87--105). Berlin, Heidelberg: Springer Berlin Heidelberg. \url{https://doi.org/10.1007/978-3-662-43711-7_9}

\leavevmode\vadjust pre{\hypertarget{ref-obbarius2021modified}{}}%
Obbarius, N., Fischer, F., Liegl, G., Obbarius, A., \& Rose, M. (2021). A modified version of the transactional stress concept according to lazarus and folkman was confirmed in a psychosomatic inpatient sample. \emph{Frontiers in Psychology}, \emph{12}, 584333.

\leavevmode\vadjust pre{\hypertarget{ref-palmer2005identifying}{}}%
Palmer, D. J., Stough, L. M., Burdenski, T. K., Jr, \& Gonzales, M. (2005). Identifying teacher expertise: An examination of researchers' decision making. \emph{Educational Psychologist}, \emph{40}(1), 13--25.

\leavevmode\vadjust pre{\hypertarget{ref-R-base}{}}%
R Core Team. (2022). \emph{R: A language and environment for statistical computing}. Vienna, Austria: R Foundation for Statistical Computing. Retrieved from \url{https://www.R-project.org/}

\leavevmode\vadjust pre{\hypertarget{ref-sachs2014physiologische}{}}%
Sachs, S. (2014). \emph{Physiologische parameter zur bewertung der lernwirksamkeit von lernsituationen} (PhD thesis). Universit{ä}t Ulm.

\leavevmode\vadjust pre{\hypertarget{ref-struyven2014teachers}{}}%
Struyven, K., \& Vanthournout, G. (2014). Teachers' exit decisions: An investigation into the reasons why newly qualified teachers fail to enter the teaching profession or why those who do enter do not continue teaching. \emph{Teaching and Teacher Education}, \emph{43}, 37--45.

\end{CSLReferences}


\end{document}
