\documentclass[]{elsarticle} %review=doublespace preprint=single 5p=2 column
%%% Begin My package additions %%%%%%%%%%%%%%%%%%%

\usepackage[hyphens]{url}


\usepackage{graphicx}
%%%%%%%%%%%%%%%% end my additions to header

\usepackage[T1]{fontenc}
\usepackage{lmodern}
\usepackage{amssymb,amsmath}
% TODO: Currently lineno needs to be loaded after amsmath because of conflict
% https://github.com/latex-lineno/lineno/issues/5
\usepackage{lineno} % add
  \linenumbers % turns line numbering on
\usepackage{ifxetex,ifluatex}
\usepackage{fixltx2e} % provides \textsubscript
% use upquote if available, for straight quotes in verbatim environments
\IfFileExists{upquote.sty}{\usepackage{upquote}}{}
\ifnum 0\ifxetex 1\fi\ifluatex 1\fi=0 % if pdftex
  \usepackage[utf8]{inputenc}
\else % if luatex or xelatex
  \usepackage{fontspec}
  \ifxetex
    \usepackage{xltxtra,xunicode}
  \fi
  \defaultfontfeatures{Mapping=tex-text,Scale=MatchLowercase}
  \newcommand{\euro}{€}
\fi
% use microtype if available
\IfFileExists{microtype.sty}{\usepackage{microtype}}{}
\usepackage[]{natbib}
\bibliographystyle{plainnat}

\ifxetex
  \usepackage[setpagesize=false, % page size defined by xetex
              unicode=false, % unicode breaks when used with xetex
              xetex]{hyperref}
\else
  \usepackage[unicode=true]{hyperref}
\fi
\hypersetup{breaklinks=true,
            bookmarks=true,
            pdfauthor={},
            pdftitle={New approaches to teachers' experience of stress: Do heart rate measurements with fitness trackers provide an efficient, inexpensive, and robust measurement method?},
            colorlinks=false,
            urlcolor=blue,
            linkcolor=magenta,
            pdfborder={0 0 0}}

\setcounter{secnumdepth}{0}
% Pandoc toggle for numbering sections (defaults to be off)
\setcounter{secnumdepth}{0}

% Pandoc syntax highlighting
\usepackage{color}
\usepackage{fancyvrb}
\newcommand{\VerbBar}{|}
\newcommand{\VERB}{\Verb[commandchars=\\\{\}]}
\DefineVerbatimEnvironment{Highlighting}{Verbatim}{commandchars=\\\{\}}
% Add ',fontsize=\small' for more characters per line
\usepackage{framed}
\definecolor{shadecolor}{RGB}{248,248,248}
\newenvironment{Shaded}{\begin{snugshade}}{\end{snugshade}}
\newcommand{\AlertTok}[1]{\textcolor[rgb]{0.94,0.16,0.16}{#1}}
\newcommand{\AnnotationTok}[1]{\textcolor[rgb]{0.56,0.35,0.01}{\textbf{\textit{#1}}}}
\newcommand{\AttributeTok}[1]{\textcolor[rgb]{0.13,0.29,0.53}{#1}}
\newcommand{\BaseNTok}[1]{\textcolor[rgb]{0.00,0.00,0.81}{#1}}
\newcommand{\BuiltInTok}[1]{#1}
\newcommand{\CharTok}[1]{\textcolor[rgb]{0.31,0.60,0.02}{#1}}
\newcommand{\CommentTok}[1]{\textcolor[rgb]{0.56,0.35,0.01}{\textit{#1}}}
\newcommand{\CommentVarTok}[1]{\textcolor[rgb]{0.56,0.35,0.01}{\textbf{\textit{#1}}}}
\newcommand{\ConstantTok}[1]{\textcolor[rgb]{0.56,0.35,0.01}{#1}}
\newcommand{\ControlFlowTok}[1]{\textcolor[rgb]{0.13,0.29,0.53}{\textbf{#1}}}
\newcommand{\DataTypeTok}[1]{\textcolor[rgb]{0.13,0.29,0.53}{#1}}
\newcommand{\DecValTok}[1]{\textcolor[rgb]{0.00,0.00,0.81}{#1}}
\newcommand{\DocumentationTok}[1]{\textcolor[rgb]{0.56,0.35,0.01}{\textbf{\textit{#1}}}}
\newcommand{\ErrorTok}[1]{\textcolor[rgb]{0.64,0.00,0.00}{\textbf{#1}}}
\newcommand{\ExtensionTok}[1]{#1}
\newcommand{\FloatTok}[1]{\textcolor[rgb]{0.00,0.00,0.81}{#1}}
\newcommand{\FunctionTok}[1]{\textcolor[rgb]{0.13,0.29,0.53}{\textbf{#1}}}
\newcommand{\ImportTok}[1]{#1}
\newcommand{\InformationTok}[1]{\textcolor[rgb]{0.56,0.35,0.01}{\textbf{\textit{#1}}}}
\newcommand{\KeywordTok}[1]{\textcolor[rgb]{0.13,0.29,0.53}{\textbf{#1}}}
\newcommand{\NormalTok}[1]{#1}
\newcommand{\OperatorTok}[1]{\textcolor[rgb]{0.81,0.36,0.00}{\textbf{#1}}}
\newcommand{\OtherTok}[1]{\textcolor[rgb]{0.56,0.35,0.01}{#1}}
\newcommand{\PreprocessorTok}[1]{\textcolor[rgb]{0.56,0.35,0.01}{\textit{#1}}}
\newcommand{\RegionMarkerTok}[1]{#1}
\newcommand{\SpecialCharTok}[1]{\textcolor[rgb]{0.81,0.36,0.00}{\textbf{#1}}}
\newcommand{\SpecialStringTok}[1]{\textcolor[rgb]{0.31,0.60,0.02}{#1}}
\newcommand{\StringTok}[1]{\textcolor[rgb]{0.31,0.60,0.02}{#1}}
\newcommand{\VariableTok}[1]{\textcolor[rgb]{0.00,0.00,0.00}{#1}}
\newcommand{\VerbatimStringTok}[1]{\textcolor[rgb]{0.31,0.60,0.02}{#1}}
\newcommand{\WarningTok}[1]{\textcolor[rgb]{0.56,0.35,0.01}{\textbf{\textit{#1}}}}

% tightlist command for lists without linebreak
\providecommand{\tightlist}{%
  \setlength{\itemsep}{0pt}\setlength{\parskip}{0pt}}







\begin{document}


\begin{frontmatter}

  \title{New approaches to teachers' experience of stress: Do heart rate
measurements with fitness trackers provide an efficient, inexpensive,
and robust measurement method?}
    \author[1]{Mandy Klatt%
  %
  }
   \ead{mandy.klatt@uni-leipzig.de} 
    \author[1]{Peer Keßler%
  %
  }
  
    \author[1,2]{Gregor Kachel%
  %
  }
  
    \author[1]{Christin Lotz%
  %
  }
  
    \author[1]{Anne Deiglmayr%
  %
  }
  
      \cortext[cor1]{Corresponding author}
  
  \begin{abstract}
  One or two sentences providing a \textbf{basic introduction} to the
  field, comprehensible to a scientist in any discipline.

  Two to three sentences of \textbf{more detailed background},
  comprehensible to scientists in related disciplines.

  One sentence clearly stating the \textbf{general problem} being
  addressed by this particular study.

  One sentence summarizing the main result (with the words
  ``\textbf{here we show}'' or their equivalent).

  Two or three sentences explaining what the \textbf{main result}
  reveals in direct comparison to what was thought to be the case
  previously, or how the main result adds to previous knowledge.

  One or two sentences to put the results into a more \textbf{general
  context}.

  Two or three sentences to provide a \textbf{broader perspective},
  readily comprehensible to a scientist in any discipline.

  XXX In this proof-of-concept study, we aimed to advance the field of
  teacher stress by collecting heart rate data with wrist-worn devices
  and testing a methodology that has the potential to provide more
  insights on the non-invasive assessment of teacher stress. XXX
  \end{abstract}
    \begin{keyword}
    heart rate; photoplethysmography; wearable electronic device;
teaching \sep 
    heart rate; photoplethysmography; wearable electronic device;
teaching
  \end{keyword}
  
 \end{frontmatter}

\begin{Shaded}
\begin{Highlighting}[]
\CommentTok{\# Seed for random number generation}
\FunctionTok{set.seed}\NormalTok{(}\DecValTok{42}\NormalTok{)}
\NormalTok{knitr}\SpecialCharTok{::}\NormalTok{opts\_chunk}\SpecialCharTok{$}\FunctionTok{set}\NormalTok{(}\AttributeTok{cache.extra =}\NormalTok{ knitr}\SpecialCharTok{::}\NormalTok{rand\_seed)}
\end{Highlighting}
\end{Shaded}

\hypertarget{introduction}{%
\section{Introduction}\label{introduction}}

Physiological data such as heart rate are becoming increasingly
important in research on stress experience. They represent an important
indicator of physical or emotional stress, as increased workload is
associated with increased heart rate \citep{sachs2014physiologische}.
Furthermore, they allow a more objective recording of stress than
self-reports \citep{rungeusing}. However, capturing heart rate in an
educational context requires the use of low-cost and non-invasive
instruments. Fitness trackers worn on the wrist have the potential to be
such a useful tool \citep{ferguson2015validity}.

To date, there is still little evidence on the usefulness of heart rate
measurements using fitness trackers in teaching and learning settings
\citep{ertzberger2016use, lowe2016educational}. \citet{rungeusing} alone
examined teacher stress in a relatively small sample (\emph{N} = 4
teachers) and showed that high heart rate indicates more stress in
teachers.

Thus, there remains a lack of robust studies on whether fitness trackers
are an efficient, low-cost, and robust measurement method for assessing
teachers' experience of arousal during teaching.

\hypertarget{theoretical-background}{%
\section{Theoretical Background}\label{theoretical-background}}

\hypertarget{stress-in-teaching-profession}{%
\subsection{Stress in Teaching
Profession}\label{stress-in-teaching-profession}}

--\textgreater{} teacher profession is one of the most stressful
professions.

Teacher stress can be defined as ``{[}\ldots{]} the experience by a
teacher of unpleasant, negative emotions, such as anger, anxiety,
tension, frustration or depression, resulting from some aspect of their
work as a teacher.'' \citep{kyriacou2001teacher}.

Teachers' individual perceptions of student misbehavior in the classroom
are closely related to their well-being \citep{ALDRUP2018126}.

--\textgreater{} wie entsteht Stress

--\textgreater{} wie wurde Stress bisher gemessen

\hypertarget{heart-rate-as-an-indicator-for-stress-or-arousal}{%
\subsection{Heart rate as an indicator for stress or
arousal}\label{heart-rate-as-an-indicator-for-stress-or-arousal}}

Heart rate is physiologically regulated by the autonomic nervous system.
An increase in the activity of the sympathetic as part of the autonomic
nervous system results in the heart rate being speeded up (``fight or
flight''). On the other hand, an increased activity of the
parasympathetic as the counterpart has the effect of slowing down the
heart rate (``rest and digest'') \citep{Battipaglia2015}. In addition to
the autonomic nervous system and genetic factors, heart rate is
influenced by numerous external factors such as social, personal,
psychological, environmental and behavioural factors
\citep{wang2022using}.

\hypertarget{wrist-worn-devices-as-a-new-approach-to-assess-physiological-measures}{%
\subsection{Wrist-worn devices as a new approach to assess physiological
measures}\label{wrist-worn-devices-as-a-new-approach-to-assess-physiological-measures}}

\citet{fuller2020reliability} showed in their review article that
wearable devices such as Fitbit watches are accurate and reliable for
measuring heart rate in controlled settings.

``The use of physiological measures enabled us to get some insight into
teachers' affective responses without disrupting the teaching process
(Mauss \& Robinson, 2009) and to reduce issues with social desirability,
retrospective bias, and high cognitive load (Becker et al., 2015; Goetz
et al., 2015; Scollon et al.,2009; Wilhelm \& Grossman, 2010). Moreover,
we found that heart rate measures discriminated between both teachers,
even when their interpersonal behavior during the lesson start was
relatively similar.''

\begin{enumerate}
\def\labelenumi{(\arabic{enumi})}
\setcounter{enumi}{19}
\tightlist
\item
  (PDF) A Quantitative Exploration of Two Teachers with Contrasting
  Emotions: Intra-Individual Process Analyses of Physiology and
  Interpersonal Behavior. Available from:
  https://www.researchgate.net/publication/329787434\_A\_Quantitative\_Exploration\_of\_Two\_Teachers\_with\_Contrasting\_Emotions\_Intra-Individual\_Process\_Analyses\_of\_Physiology\_and\_Interpersonal\_Behavior
  {[}accessed Dec 07 2022{]}.
\end{enumerate}

\hypertarget{aim-of-the-study}{%
\subsection{Aim of the study}\label{aim-of-the-study}}

In the present study, we assessed HR measures and self-report data of
pre- and in-service teachers in a controlled teaching-learning setting.
The aim was to investigate whether heart rate measurements using
wrist-worn fitness trackers are a suitable and effective method
\textbf{(1)} to map differences in states of arousal between five
different phases (pre-teaching phase, teaching phase, post-teaching
phase, interview phase and end phase) and \textbf{(2)} to evaluate the
correlation between self-reported evaluations and HR measures.

\textbf{(H1)} We expected heart rates to be higher during the teaching
phase than during the pre- and the three post-teaching phases, and that
the HR measures would decrease over the course of the study.
\textbf{(H2)} We also predicted that HR and a high ranking on the
negative scale on our survey (feeling disturbed by disruptions) would
correlate positively and a high ranking on the positive scale (feeling
confident in dealing with disruptions) would follow the inverse pattern.

\hypertarget{methods}{%
\section{Methods}\label{methods}}

We report how we determined our sample size, all data exclusions (if
any), all manipulations, and all measures in the study.

\hypertarget{participants}{%
\subsection{Participants}\label{participants}}

\bibliography{r-references.bib}


\end{document}
