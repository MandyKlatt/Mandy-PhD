\documentclass[preprint, 3p,
authoryear]{elsarticle} %review=doublespace preprint=single 5p=2 column
%%% Begin My package additions %%%%%%%%%%%%%%%%%%%

\usepackage[hyphens]{url}

  \journal{Computers \& Education} % Sets Journal name

\usepackage{graphicx}
%%%%%%%%%%%%%%%% end my additions to header

\usepackage[T1]{fontenc}
\usepackage{lmodern}
\usepackage{amssymb,amsmath}
% TODO: Currently lineno needs to be loaded after amsmath because of conflict
% https://github.com/latex-lineno/lineno/issues/5
\usepackage{lineno} % add
\usepackage{ifxetex,ifluatex}
\usepackage{fixltx2e} % provides \textsubscript
% use upquote if available, for straight quotes in verbatim environments
\IfFileExists{upquote.sty}{\usepackage{upquote}}{}
\ifnum 0\ifxetex 1\fi\ifluatex 1\fi=0 % if pdftex
  \usepackage[utf8]{inputenc}
\else % if luatex or xelatex
  \usepackage{fontspec}
  \ifxetex
    \usepackage{xltxtra,xunicode}
  \fi
  \defaultfontfeatures{Mapping=tex-text,Scale=MatchLowercase}
  \newcommand{\euro}{€}
\fi
% use microtype if available
\IfFileExists{microtype.sty}{\usepackage{microtype}}{}

\ifxetex
  \usepackage[setpagesize=false, % page size defined by xetex
              unicode=false, % unicode breaks when used with xetex
              xetex]{hyperref}
\else
  \usepackage[unicode=true]{hyperref}
\fi
\hypersetup{breaklinks=true,
            bookmarks=true,
            pdfauthor={},
            pdftitle={New approaches to teachers' experience of stress: Do heart rate measurements with fitness trackers provide an efficient, inexpensive, and robust measurement method?},
            colorlinks=false,
            urlcolor=blue,
            linkcolor=magenta,
            pdfborder={0 0 0}}

\setcounter{secnumdepth}{5}
% Pandoc toggle for numbering sections (defaults to be off)


% tightlist command for lists without linebreak
\providecommand{\tightlist}{%
  \setlength{\itemsep}{0pt}\setlength{\parskip}{0pt}}


% Pandoc citation processing
\newlength{\cslhangindent}
\setlength{\cslhangindent}{1.5em}
\newlength{\csllabelwidth}
\setlength{\csllabelwidth}{3em}
\newlength{\cslentryspacingunit} % times entry-spacing
\setlength{\cslentryspacingunit}{\parskip}
% for Pandoc 2.8 to 2.10.1
\newenvironment{cslreferences}%
  {}%
  {\par}
% For Pandoc 2.11+
\newenvironment{CSLReferences}[2] % #1 hanging-ident, #2 entry spacing
 {% don't indent paragraphs
  \setlength{\parindent}{0pt}
  % turn on hanging indent if param 1 is 1
  \ifodd #1
  \let\oldpar\par
  \def\par{\hangindent=\cslhangindent\oldpar}
  \fi
  % set entry spacing
  \setlength{\parskip}{#2\cslentryspacingunit}
 }%
 {}
\usepackage{calc}
\newcommand{\CSLBlock}[1]{#1\hfill\break}
\newcommand{\CSLLeftMargin}[1]{\parbox[t]{\csllabelwidth}{#1}}
\newcommand{\CSLRightInline}[1]{\parbox[t]{\linewidth - \csllabelwidth}{#1}\break}
\newcommand{\CSLIndent}[1]{\hspace{\cslhangindent}#1}





\begin{document}


\begin{frontmatter}

  \title{New approaches to teachers' experience of stress: Do heart rate
measurements with fitness trackers provide an efficient, inexpensive,
and robust measurement method?}
    \author[Leipzig University]{Mandy Klatt%
  \corref{cor1}%
  \fnref{1}}
   \ead{mandy.klatt@uni-leipzig.de} 
    \author[Leipzig University]{Peer Keßler%
  %
  \fnref{2}}
   \ead{Peer@example.com} 
    \author[Leipzig University, Max Planck Institute for Evolutionary
Anthropology]{Gregor Kachel%
  %
  \fnref{1}}
   \ead{gregor@example.com} 
    \author[Leipzig University]{Christin Lotz%
  %
  \fnref{1}}
   \ead{christin@example.com} 
    \author[Leipzig University]{Anne Deiglmayr%
  %
  \fnref{3}}
   \ead{anne@example.com} 
      \affiliation[Leipzig University]{
    organization={Empirische Schul- und
Unterrichtsforschung},addressline={Marschnerstr.
29},city={Leipzig},postcode={4109},state={Sachsen},country={Germany},}
    \affiliation[Max Planck Institute for Evolutionary Anthropology]{
    organization={Department},addressline={Deutscher Pl.
6},city={Leipzig},postcode={4103},state={Sachsen},country={Germany},}
    \cortext[cor1]{Corresponding author}
    \fntext[1]{ConceptualizationWriting - Original Draft
PreparationWriting - Review \& Editing}
    \fntext[2]{Writing - Original Draft PreparationWriting - Review \&
Editing}
    \fntext[3]{Supervision}
  
  \begin{abstract}
  One or two sentences providing a \textbf{basic introduction} to the
  field, comprehensible to a scientist in any discipline.

  Two to three sentences of \textbf{more detailed background},
  comprehensible to scientists in related disciplines.

  One sentence clearly stating the \textbf{general problem} being
  addressed by this particular study.

  One sentence summarizing the main result (with the words
  ``\textbf{here we show}'' or their equivalent).

  Two or three sentences explaining what the \textbf{main result}
  reveals in direct comparison to what was thought to be the case
  previously, or how the main result adds to previous knowledge.

  One or two sentences to put the results into a more \textbf{general
  context}.

  Two or three sentences to provide a \textbf{broader perspective},
  readily comprehensible to a scientist in any discipline.

  XXX In this proof-of-concept study, we aimed to advance the field of
  teacher stress by collecting heart rate data with wrist-worn devices
  and testing a methodology that has the potential to provide more
  insights on the non-invasive assessment of teacher stress. XXX
  \end{abstract}
    \begin{keyword}
    heart rate \sep photoplethysmography \sep wearable electronic
device \sep 
    teaching
  \end{keyword}
  
 \end{frontmatter}

\hypertarget{present-investigation}{%
\section{Present investigation}\label{present-investigation}}

The aim was to investigate whether HR measures assessed by a Fitbit
Charge 4 are a suitable and effective method to \textbf{(1)} map
teachers course of arousal over the course of a five phase lab study,
including a micro teaching unit, and \textbf{(2)} examine whether HR
measures can be predicted by self-reported data on cognitive appraisal.

Within the time frame of approximately two hours, we investigated five
intervals with a duration of 10-minutes each: In the (1) pre-teaching
phase, the subjects were prepared for the following micro teaching unit
teaching lesson and familiarized with the setting. During the (2)
teaching phase, the participants taught a 15-minute self-prepared lesson
to a ``class'' of three actors that simulated nine classroom disruptions
. In the following (3) post-teaching phase, the subjects answered a
questionnaires, followed by and in the (4) interview phase, in which
they subjects watched thea pre-recorded video of their 15-minute lesson
to assess the self-report data of how disrupted subjects felt and how
confident they felt in dealing with disruptions. In the (5) end phase,
the subject answered another second questionnaire.

According to previous findings that fitness trackers can be used as a
low-cost, non-invasive method of measuring HR (\textbf{hajj2022wrist?};
Fuller et al. 2020) and that different HRs of teachers can be measured
in different teaching phases (\textbf{donker2020associations?};
\textbf{junker2021potential?}), we formulate the hypotheses as follows:

\textbf{Hypothesis 1}. In a first step, we wanted to display
exploratively the course of HR during the entire study. Additionally, we
presumed the highest mean HR in the (2) teaching phase and lower mean
values in all other phases (\textbf{Hypothesis 1a}). Moreover, we
expected an increase in HR in the (1) pre-teaching phase as the first
phase and a decrease in the following phases (\textbf{Hypothesis 1b}).
\textbf{Hypothesis 2}. We statistically predicted the subjects'
standardized mean HR for the (2) teaching, the (3) post-teaching, the
(4) interview and the (5) end phase with teaching experience and
self-repoted data on \ldots. With respect to teaching experience, Wwe
expected a lower HR in teachers with more teaching experience for the
four phases (\textbf{Hypothesis 2a}). According to the relationship
between physiological arousal and cognitive appraisal, we controlled for
shared variance with the self-reported data. Concerning the \ldots, Wwe
expected higher HR values for for the four phases that teachers who
reported that they felt more disrupted by disruptions would have higher
standardized mean HR (\textbf{Hypotheses 2b}). In contrast, individuals
with high confidence in dealing with disruptions would have a lower
standardized mean HR in the four phases (\textbf{Hypothesis 2c}). When
considering the three predictors in concert and controlling for their
common variance, we expected teaching experience and self-reported data
to remain substantial predictors (\textbf{Hypothesis 2d}).

\hypertarget{references}{%
\section*{References}\label{references}}
\addcontentsline{toc}{section}{References}

\hypertarget{refs}{}
\begin{CSLReferences}{1}{0}
\leavevmode\vadjust pre{\hypertarget{ref-fuller2020reliability}{}}%
Fuller, Daniel, Emily Colwell, Jonathan Low, Kassia Orychock, Melissa
Ann Tobin, Bo Simango, Richard Buote, et al. 2020. {``Reliability and
Validity of Commercially Available Wearable Devices for Measuring Steps,
Energy Expenditure, and Heart Rate: Systematic Review.''} \emph{JMIR
mHealth and uHealth} 8 (9): e18694.

\end{CSLReferences}


\end{document}
