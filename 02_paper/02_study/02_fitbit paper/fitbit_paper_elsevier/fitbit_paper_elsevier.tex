\documentclass[man]{elsarticle} %review=doublespace preprint=single 5p=2 column
%%% Begin My package additions %%%%%%%%%%%%%%%%%%%

\usepackage[hyphens]{url}


\usepackage{graphicx}
%%%%%%%%%%%%%%%% end my additions to header

\usepackage[T1]{fontenc}
\usepackage{lmodern}
\usepackage{amssymb,amsmath}
% TODO: Currently lineno needs to be loaded after amsmath because of conflict
% https://github.com/latex-lineno/lineno/issues/5
\usepackage{lineno} % add
  \linenumbers % turns line numbering on
\usepackage{ifxetex,ifluatex}
\usepackage{fixltx2e} % provides \textsubscript
% use upquote if available, for straight quotes in verbatim environments
\IfFileExists{upquote.sty}{\usepackage{upquote}}{}
\ifnum 0\ifxetex 1\fi\ifluatex 1\fi=0 % if pdftex
  \usepackage[utf8]{inputenc}
\else % if luatex or xelatex
  \usepackage{fontspec}
  \ifxetex
    \usepackage{xltxtra,xunicode}
  \fi
  \defaultfontfeatures{Mapping=tex-text,Scale=MatchLowercase}
  \newcommand{\euro}{€}
\fi
% use microtype if available
\IfFileExists{microtype.sty}{\usepackage{microtype}}{}
\usepackage[]{natbib}
\bibliographystyle{plainnat}

\ifxetex
  \usepackage[setpagesize=false, % page size defined by xetex
              unicode=false, % unicode breaks when used with xetex
              xetex]{hyperref}
\else
  \usepackage[unicode=true]{hyperref}
\fi
\hypersetup{breaklinks=true,
            bookmarks=true,
            pdfauthor={},
            pdftitle={New approaches to teachers' experience of stress: Do heart rate measurements with fitness trackers provide an efficient, inexpensive, and robust measurement method?},
            colorlinks=false,
            urlcolor=blue,
            linkcolor=magenta,
            pdfborder={0 0 0}}

\setcounter{secnumdepth}{0}
% Pandoc toggle for numbering sections (defaults to be off)
\setcounter{secnumdepth}{0}


% tightlist command for lists without linebreak
\providecommand{\tightlist}{%
  \setlength{\itemsep}{0pt}\setlength{\parskip}{0pt}}







\begin{document}


\begin{frontmatter}

  \title{New approaches to teachers' experience of stress: Do heart rate
measurements with fitness trackers provide an efficient, inexpensive,
and robust measurement method?}
    \author[1]{Mandy Klatt%
  %
  }
   \ead{mandy.klatt@uni-leipzig.de} 
    \author[1]{Peer Keßler%
  %
  }
  
    \author[1,2]{Gregor Kachel%
  %
  }
  
    \author[1]{Christin Lotz%
  %
  }
  
    \author[1]{Anne Deiglmayr%
  %
  }
  
      \cortext[cor1]{Corresponding author}
  
  \begin{abstract}
  One or two sentences providing a \textbf{basic introduction} to the
  field, comprehensible to a scientist in any discipline.

  Two to three sentences of \textbf{more detailed background},
  comprehensible to scientists in related disciplines.

  One sentence clearly stating the \textbf{general problem} being
  addressed by this particular study.

  One sentence summarizing the main result (with the words
  ``\textbf{here we show}'' or their equivalent).

  Two or three sentences explaining what the \textbf{main result}
  reveals in direct comparison to what was thought to be the case
  previously, or how the main result adds to previous knowledge.

  One or two sentences to put the results into a more \textbf{general
  context}.

  Two or three sentences to provide a \textbf{broader perspective},
  readily comprehensible to a scientist in any discipline.

  XXX In this proof-of-concept study, we aimed to advance the field of
  teacher stress by collecting heart rate data with wrist-worn devices
  and testing a methodology that has the potential to provide more
  insights on the non-invasive assessment of teacher stress. XXX
  \end{abstract}
    \begin{keyword}
    heart rate; photoplethysmography; wearable electronic device;
teaching \sep 
    heart rate; photoplethysmography; wearable electronic device;
teaching
  \end{keyword}
  
 \end{frontmatter}



\bibliography{r-references.bib}


\end{document}
